\chapter{Fazit und Ausblick}

Wie die Untersuchung der variablen Nachbarschaftssuche für das Capacitated Facility Location Problem zeigt, variiert die Lösungsgüte in Abhängigkeit der eingesetzten Teilprozeduren und Parameter.
Zu sehen ist vor allem der Unterschied der Lösungsgüte in Abhängigkeit der Initial-Lösung, der Nachbarschaftsdistanz  und der lokalen Suche.
Die \ac{RVNS} dominiert den Vergleich der zur Auswahl stehenden Initial-Verfahren.
Sie stellt einen Kompromiss zwischen der Güte der Lösung und der Rechenzeit unter Berücksichtigung der zugehörigen Kosten dar.
Aufgrund der eingesetzten maximalen Nachbarschaftsdistanzen $k_{max} = 3$ können die Nachbarschaftsverfahren einfach aus schlechten lokalen Minima entkommen und neue Lösungen generieren.
Das lokale Suchverfahren $Best$ $Improvement$ ist ebenfalls ein fester Teil der Ergebnisfindung, um qualitativ hochwertige lokale Minima zu finden.
Die Verfahren zur Generierung von Nachbarschaftslösungen stellen sich im Allgemeinen als ähnlich gut heraus, wobei das Zufallsverfahren $ShakingA$ als geringfügig besser zu bewerten ist.
Die Idee eine Nachbarschaftslösung anhand der Kundenzuweisungen oder der entstehenden Transportkosten zu erzeugen, stellt sich dennoch bei passender Nachbarschaftsdistanz als gute Alternative heraus.
Die Berechnung der Kundenzuordnungen einer gegebenen Standortverteilung ist ein wichtiger Bestandteil der Ergebnisfindung.
Da diese mit der optimierenden MODI-Methode berechnet werden, können die zugehörigen Lösungen im Gegensatz zu einem Greedy-Verfahren aussagekräftig und valide verglichen werden.
\\
Die variable Nachbarschaftssuche wird den Anforderungen an eine Heuristik im Bezug auf eine reduzierte Rechenzeit gegenüber exakten Verfahren gerecht.
Die Rechenergebnisse bestätigen, dass eine Ergebnisfindung in einem kurzen Zeitintervall deutlich besser skaliert, als bei einem exakten Verfahren und das Finden guter lokaler Minima wird ebenfalls erfüllt. 
Noch größere Probleminstanzen können aussagekräftig in kurzer Zeit berechnet werden und Erweiterungen des Facility Location Problems können ebenfalls in die vorliegende Implementierung eingebettet werden.
Eine Reduzierung der Rechenzeit kann vor allem bei einem verbesserten Verfahren zum Bestimmen der MODI-Ausgangslösung erzielt werden, um anschließend noch mehr Teile des Lösungsraums zu betrachten.