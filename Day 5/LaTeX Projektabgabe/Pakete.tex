%Packages
\usepackage{babel}
\usepackage[parfill]{parskip}
\usepackage{mathptmx} 
\usepackage[utf8]{inputenc}
\usepackage[T1]{fontenc}
\usepackage[intlimits]{amsmath}
\usepackage{amssymb}
\usepackage{url}
\usepackage{subcaption}
\usepackage{geometry}
\usepackage{multirow}
\usepackage{booktabs}
\usepackage[page]{totalcount}
\usepackage[dvipsnames,svgnames,x11names,table]{xcolor}
\usepackage{scrlayer-scrpage}
\usepackage[parfill]{parskip}
\usepackage{chngcntr}
\usepackage{mathpazo}
\usepackage{setspace}
\usepackage[]{ragged2e}
\usepackage[printonlyused]{acronym}
\usepackage{etoolbox}
\usepackage[toc,page]{appendix} 
\usepackage[final]{pdfpages}
\usepackage{listingsutf8}
\usepackage{setspace}
\usepackage{listings}
\usepackage[justification=centering, singlelinecheck=false]{caption}
\usepackage[acronym, xindy]{glossaries}
\usepackage[noend]{algpseudocode}
\usepackage{forloop}
\usepackage{algorithm}
\usepackage{varwidth}
\usepackage[hidelinks]{hyperref}
\hypersetup{allcolors=black}
\usepackage{makecell}
\usepackage[final]{listofsymbols}
\usepackage{setspace}

%%%% START DEFINES %%%%

% BIBLIO
\usepackage[
style=authoryear,
isbn=false,                
doi=false,
backend=bibtex
]{biblatex}
\setlength{\bibitemsep}{1em}     % Abstand zwischen den Literaturangaben
\setlength{\bibhang}{2em}        % Einzug nach jeweils erster Zeile

% Symbolverzeichnis
\renewcommand{\symheadingname}{Symbolverzeichnis}
\newcommand\mysymbol[3]{%
\protected\gdef#1{#2}%
\item[$#2$]#3}

% Counter Dokumentenweit
\counterwithout{figure}{chapter}
\counterwithout{table}{chapter} 

% Silbentrennung
\tolerance=1
\emergencystretch=\maxdimen
\hyphenpenalty=10000
\hbadness=10000

% Caption Centering
\DeclareCaptionFormat{myformat}{%
  \begin{varwidth}{\linewidth}%
    \centering
    #1#2#3%
  \end{varwidth}%
}
\captionsetup{format=myformat}% global activation

% Farben
\definecolor{rubgray}{HTML}{F3F3F3}
\definecolor{rubgray2}{gray}{0.8}
\definecolor{rubblue}{HTML}{003560}
\definecolor{rubgreen}{RGB}{141,174,16}

% Geometrie
 \geometry{
 	left=3cm,
  	right=2.5cm,
  	top=2.5cm,
  	bottom=2.0cm,
    total={170mm,257mm}
 }

%% footer
\setlength\footheight{45pt}
\clearpairofpagestyles
\cfoot*{\textrm{\thepage}}
\setkomafont{pageheadfoot}{\textrm\tiny}
\addtokomafont{pagenumber}{\textrm\tiny}

%Pages counter
\renewcommand\pagemark{{%
    \textbf{\MakeUppercase{\pagename}}
    \usekomafont{pagenumber}%
    {\thepage} | \totalpages
}}

%% Grauer Kasten
\DeclareNewLayer[
    background,
    bottommargin,
    addvoffset=-\footheight,
    addvoffset=-.5ex,
    mode=picture,
    contents=\putUL{\textcolor{rubgray}{\rule[-\layerheight]{\layerwidth}{\layerheight}}}
]{bottomrule}
\DeclareNewLayer[
    clone=bottomrule,
    textarea,
    voffset=0pt,
    height=\paperheight,
    addhoffset=\textwidth,
    addhoffset=5ex
]{rightrule}

% Titleseite Style
\DeclarePageStyleByLayers{titlepage}{}
\AddLayersToPageStyle{titlepage}{bottomrule,rightrule}
\usepackage{graphicx}
\DeclareNewLayer[
    clone=rightrule,
    contents=\putUL{\raisebox{-\height}{\makebox[5mm][r]{\includegraphics[scale=.7]{Bilder/logo-rub-102.jpg}}}}
]{titlepagelogo}
\AddLayersToPageStyle{titlepage}{titlepagelogo}

% KOMA Ebenen Farben 
\setkomafont{chapter}{\normalfont\LARGE\color{rubblue}}
\addtokomafont{section}{\normalfont\Large\color{rubgreen}}
\addtokomafont{subsection}{\normalfont\large\color{rubblue}}
\addtokomafont{subsubsection}{\normalfont\large\color{rubgreen}}
\renewcommand{\familydefault}{\rmdefault}

% Stops the reset of the footnode number on each chapter
\counterwithout{footnote}{chapter}
% 1,5 facher WORD abstand
\linespread{1.25}

% Itemize Kasten in RubStyle
\newcommand{\localtextbulletone}{\textcolor{rubblue}{\raisebox{0.45ex}{\rule{1ex}{1ex}}}}
\renewcommand{\labelitemi}{\localtextbulletone}
% Modify chapter to stop doing a pagebreak
\makeatletter
\patchcmd{\scr@startchapter}{\if@openright\cleardoublepage\else\clearpage\fi}{}{}{}
\makeatother

% Listings
\setlength{\parindent}{0pt}
\definecolor{listinggray}{gray}{0.9}
\definecolor{lbcolor}{RGB}{247,247,247}
\definecolor{Darkgreen}{RGB}{70, 160, 64}
\definecolor{LightLime}{rgb}{0.3,0.5,0.4}

\lstset{
backgroundcolor=\color{lbcolor},
    	tabsize=1,    
        basicstyle=\scriptsize,
        upquote=true,
        aboveskip={0\baselineskip},
        columns=fixed,
        showstringspaces=false,
        extendedchars=false,
        breaklines=true,
        prebreak = \raisebox{0ex}[0ex][0ex]{\ensuremath{\hookleftarrow}},
        keywords=[2]{string, vector},
		keywordstyle=[2]{\color{LightLime}},
		keywordstyle=\color[rgb]{0,0,1},
        commentstyle=\color[rgb]{0.026,0.112,0.095},
        stringstyle=\color[rgb]{0.627,0.126,0.941},
        numberstyle=\color[rgb]{0.205, 0.142, 0.73},
        frame=single,
        numbers=none,
        showtabs=false,
        showspaces=false,
        showstringspaces=false,
        identifierstyle=\ttfamily
}

% Appendix Nummerierung der Chapters
\newcommand{\nocontentsline}[3]{}
\newcommand{\tocless}[2]{\bgroup\let\addcontentsline=\nocontentsline#1{#2}\egroup}

% LoF neu definieren
\makeatletter
\AtEndEnvironment{figure}{\gdef\there@is@a@figure{}} 
\AtEndDocument{\ifdefined\there@is@a@figure\label{fig:was:used:in:doc}\fi} 
\newcommand{\LoF}{\@ifundefined{r@fig:was:used:in:doc}{}{\listoffigures}}%
\makeatother

% LoT neu definieren
\makeatletter
\renewcommand\tableofcontents{%
    \section*{\contentsname
        \@mkboth{%
           \MakeUppercase\contentsname}{\MakeUppercase\contentsname}}%
    \global\@printlisttrue
    \@starttoc{toc}%
    }
\renewcommand\listoftables{%
  \check@list{lot}
  \if@printlist
    \chapter*{\listtablename}%
      \@mkboth{%
          \MakeUppercase\listtablename}%
         {\MakeUppercase\listtablename}%
  \fi
  \@starttoc{lot}%
}
\def\@starttoc#1{%
  \begingroup
    \makeatletter
    \if@printlist
      \@input{\jobname.#1}%
    \fi
    \if@filesw
      \expandafter\newwrite\csname tf@#1\endcsname
      \immediate\openout \csname tf@#1\endcsname \jobname.#1\relax
    \fi
    \@nobreakfalse
  \endgroup}
\newif\if@printlist
\def\check@list#1{%
  \global\@printlistfalse
  \setbox\z@=\vbox{\makeatletter\@input{\jobname.#1}}%
  \ifdim\ht\z@>\z@\global\@printlisttrue\fi}
\makeatother

% List of Content
\renewcommand{\contentsname}{\sffamily\textcolor{red}{Another title for the table of contents}}

% Algorithm Define
\makeatletter
\def\BState{\State\hskip-\ALG@thistlm}
\makeatother
\newcommand{\INDSTATE}[1][1]{\STATE\hspace{#1\algorithmicindent}}
\newcounter{ct}
\newcommand{\markdent}[1]{\forloop{ct}{0}{\value{ct} < #1}{\hspace{\algorithmicindent}}}
\newcommand{\markcomment}[1]{\Statex\markdent{#1}}
\algnewcommand{\Initialize}[1]{%
  \State \textbf{Initialisierung:}
  \Statex \hspace*{\algorithmicindent}\parbox[t]{.8\linewidth}{\raggedright #1}
}
\algnewcommand{\Prozedur}[1]{%
  \State \textbf{Prozedur:}
  \Statex \hspace*{\algorithmicindent}\parbox[t]{.8\linewidth}{\raggedright #1}
}

% Keywords Definition
\providecommand{\keywords}[1]{\textbf{\textit{Keywords -}} #1}

% LINEBREAK IN TABLE
\renewcommand\theadalign{cb}
\renewcommand\theadfont{\bfseries}
\renewcommand\theadgape{\Gape[4pt]}
\renewcommand\cellgape{\Gape[4pt]}

% VECTOR 
\newcommand{\myvec}[1]{\ensuremath{\begin{pmatrix}#1\end{pmatrix}}}